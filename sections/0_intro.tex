\documentclass[../main.tex]{subfiles}

\begin{document}

\chapter*{Вступ}
\addcontentsline{toc}{chapter}{Вступ}

В наш час важко уявити світ без інформаційних технологій. Програмне забезпечення оточує нас усюди — від стільникового телефону до системи управління штучним серцем. Такий широкий спектр можливих застосувань програмного забезпечення, а~також дуже висока швидкість впровадження нових технологій призводять до неминучої появи дефектів у програмному забезпеченні. Дефекти можуть бути різного ступеня небезпечності — від некритичних, до нагальних. Саме тому існують системи відслідковування проблем програмного забезпечення. Такі системи дозволяють зберігати інформацію щодо проблем або пропозицій з покращення системи, сторювати на основі цієї інформації задачі, що потребують вирішення, групувати та пріоритезувати їх. Таким чином, на~даний момент системи відслідковування проблем є необхідною складовою процесу розробки будь-якого програмного продукту.

Багато багтрекерів, зокрема ті, що використовуються більшістю проектів відкритого програмного забезпечення, дозволяють користувачам вводити звіт про помилку безпосередньо. Інші системи використовуються лише всередині компаній чи організацій, що займаються розробкою програмного забезпечення. Здебільшого багтрекери використовуються спільно з іншими програмами керування проектами програмного забезпечення.

Наявність багтрекера вкрай важлива у розробці програмного забезпечення, і вони широко використовуються компаніями, що розробляють програмні продукти. Послідовне використання багтрекера вважається однією з «ознак хорошої команди програмістів»~\cite{painless_bug_tracking}.

Головна задача багтрекера полягає в забезпеченні централізованого огляду запитів розробки та їх стану. Список пріоритетів незавершених пунктів забезпечує вагомий внесок при визначенні плану продукту.

Об'єктом даної дипломної роботи є методи та засоби розробки розподілених web-систем. Предметом є методи та засоби розробки систем відслідковування проблем програмного забезпечення з відкритим API.

Метою даної роботи є створення розподіленої системи відслідковування проблем ПЗ з відкритим API та розробка Android клієнту цієї системи. Майбутній продукт повинен задовольняти такі вимоги:
\begin{enumerate}
\item Розповсюджуватись на основі вільної ліцензії.
\item Мати можливість швидкого розгортання на сервері користувача.
\item Надавати API для взаемодії з клієнтами під будь-яку платформу.
\item Мати готову реалізацію клієнту під платформу Android.
\item Надавати можливість створювати/редагувати/видаляти звіти щодо багів/пропозицій.
\item Надавати можливість відстежувати зміни стану багів/пропозицій.
\item Надавати можливість назначати відповідального за баг/пропозицію.
\item Мати систему керування ролями користувачів в межах проекту.
\end{enumerate}

Окрім ліцензійних аспектів та імплементації у вигляді одного публічного API і клієнтів під різні платформи, однією з важливих відмінностей майбутнього програмного продукту від існуючих аналогів є можливість присвоєння більш ніж одного відповідального за окрему задачу робітника. Доцільність такого рішення продиктована існуванням задач, котрі не можуть бути вирішені ефективно лише однією людиною.

Дипломна робота складається із вступу, чотирьох розділів, висновків, списку використаної літератури (23 найменування), 12 додатків. Робота містить \totaltables\ таблиць, \totalfigures\  рисунків.

Роботу викладено у вільному доступі у вигляді Git репозиторіїв, посилання: \url{https://github.com/s0nerik/BugTracktor}, \url{https://github.com/s0nerik/BugTracktorAndroid}.

\end{document}