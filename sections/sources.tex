\chapter*{Список використаних джерел}
\addcontentsline{toc}{chapter}{Список використаних джерел}

\begin{enumerate}
	\item Swagger. Getting started. [Електронний ресурс] – Режим доступу до ресурсу: http://swagger.io/getting-started (дата звернення 24.03.2017) – Назва з екрана.
	\item Pattern-Oriented Software Architecture. / [F. Buschmann, R. Meunier, H. Rohnert та ін.]. – Нью-Джерсі: Wiley, 1996. – 476 с.
	\item Gebski S. Jesus-Driven Development [Електронний ресурс]~/~Sebastian~Gebski – Режим доступу до ресурсу: http://no-kill-switch.ghost.io/jesus-driven-development (дата звернення 25.03.2017) – Назва з екрана.
	\item GitHub.~Web~Application~Frameworks.~[Електронний~ресурс]~–~Режим доступу~до~ресурсу:~https://github.com/showcases/web-application-frameworks~(дата~звернення~16.03.2017)~–~Назва~з~екрана.
	\item Google. Material design. [Електронний ресурс] – Режим доступу до ресурсу: https://material.io/guidelines.
	\item Spolsky J. Painless Bug Tracking [Електронний ресурс]~/~Joel~Spolsky~–~Режим~доступу~до~ресурсу: https://www.joelonsoftware.com/2000/11/08/painless-bug-tracking (дата звернення 16.03.2017) – Назва з екрана.
	\item StatCounter. Desktop vs Mobile Market Share Worldwide [Електронний ресурс] – Режим доступу до ресурсу: http://gs.statcounter.com/platform-market-share/desktop-mobile/worldwide/\#monthly-201204-201704 (дата звернення 20.03.2017) – Назва з екрана.
	\item Weisfeld M. The Object-Oriented Thought Process / Matt Weisfeld. – Бостон: Addison-Wesley Professional, 2013. – 336 с.
	\item RSS [Електронний ресурс] – Режим доступу до ресурсу: https://en.wikipedia.org/wiki/RSS.
	\item Version control systems [Електронний ресурс] – Режим доступу до ресурсу:~https://en.wikipedia.org/wiki/Version\_control (дата звернення 15.03.2017) – Назва з екрана.
	\item Система відстеження помилок [Електронний~ресурс] – Режим доступу до~ресурсу:~ https://uk.wikipedia.org/wiki/Система\_відстеження\_помилок (дата звернення 15.03.2017) – Назва з екрана.
	\item Авраменко В. С. Проектування інформаційних систем / В. С. Авраменко, С. В. Голуб, В. І. Салапатов. – Черкаси: ЧНУ ім. Богдана Хмельницького, 2015. – 496 с.
	\item Объектно-ориентированный анализ и проектирование с примерами приложений. / [Г. Буч, Р. А. Максимчук, М. У. Энгл та ін.]. – М.: Вильямс, 2008. – 720 с.
	\item Грэхем И. Объектно-ориентированные методы. Принципы и практика. / Иан Грэхем. – М.: Вильямс, 2004. – 880 с.
	\item Влиссидес Д. Применение шаблонов проектирования. Дополнительные штрихи. / Дж. Влиссидес. – М.: Вильямс, 2003. – 144 с.
	\item Канер К. Тестирование программного обеспечения. Фундаментальные концепции менеджмента бизнес-приложений. / К. Канер, Д. Фолк, К. Е. Нгуен. – Киев: ДиаСофт, 2001. – 544 с.
	\item Криспин Л. Гибкое тестирование: практическое руководство для тестировщиков ПО и гибких команд. / Л. Криспин, Д. Грегори. – М.: Вильямс, 2010. – 464 с.
	\item Ларман К. Применение UML 2.0 и шаблонов проектирования. / Крэг Ларман. – М.: Вильямс, 2006. – 736 с.
	\item Калбертсон Р. Быстрое тестирование. / Р. Калбертсон, К. Браун, Г. Кобб. – М.: Вильямс, 2002. – 374 с.
	\item Alexander С. A Pattern Language: Towns, Buildings, Construction. / С. Alexander, S. Ishikawa, M. Silverstein M.. // Oxford University Press. – 1977.
	\item Смит Д. М. Элементарные шаблоны проектирования. / Джейсон Мак-Колм Смит. – М.: Вильямс, 2012. – 304 с.
	\item Шаллоуей А. Шаблоны проектирования. Новый подход к объектно-ориентированному анализу и проектированиюи. / А. Шаллоуей, Д. Р. Тротт. – М.: Вильямс, 2002. – 288 с.
	\item Приемы объектно-ориентированного проектирования. Паттерны проектирования. / Э.Гамма, Р. Хелм, Р. Джонсон, Д. Влиссидес. – Спб.: Питер, 2001. – 368 с.
\end{enumerate}