\documentclass[../main.tex]{subfiles}

\begin{document}

\chapter{Теоретичні відомості та аналітика}

\section{Призначення та область застосування багтрекерів}

Розробка будь-якої системи є трудомістким та важким процесом, що супроводжується високим ризиком допущення помилки на всіх етапах. Тому, зазвичай, паралельно в розробкою системи відбувається і її тестування на предмет нових дефектів та регресій.

Для того, щоб дефект був усунутий, інформація про нього повинна дістатися до розробника, котрий зможе модифікувати код відповідним чином.

Система відстеження помилок у найпростішому варіанті — це процес, що включає в себе виявлення помилки, її опис, виправлення і перевірку цього виправлення, тобто процес «стеження» за багом протягом всього як його життєвого циклу, так і життєвого циклу розробки в цілому.\cite{bugtracking_systems}

Багтрекери є підвидом таск-менеджерів. Основною відмінністю багтрекерів від таск-менеджерів є  орієнтованість на покращення ефективності вирішення проблем, що зазвичай виявляється в наданні додаткового функціоналу, пов'язаного з прикладною областю. Наприклад, для будь-якого програмного продукту досить важливим засобом пошуку проблеми є stack trace помилки, тому наявність підтримки прикріплення stack trace'у до помилки є досить суттєвим покращенням порівняно зі звичайним таск-менеджером щодо аналізу проблеми, що виникла під час роботи продукту. Для ілюстрації можливого засобу поліпшення ефективності вирішення проблем в конкретній прикладній області програмування можна навести приклад з застосунком, основною задачею котрого є відстеження локацій користувача задля передбачення можливих перешкод на шляху руху автомобіля — у випадку цієї прикладної області було б доцільно мати можливість прикріплення звіту щодо отриманих географічних даних з самого застосунку до основного звіту щодо проблеми, оскільки ця інформація є необхідною для вирішення всіх проблем, пов'язаних з неправильним трактуванням поточних доріг, прилеглих перешкод, тощо.

Головний компонент такої системи — база даних, що містить відомості про виявлені дефекти. Ці відомості можуть включати в себе:
\begin{enumerate}
	\item Номер (ідентифікатор) дефекту.
	\item Хто повідомив про дефект.
	\item Дата і час виявлення дефекту.
	\item Версія продукту, в якій виявлено дефект.
	\item Серйозність (критичність) дефекту та пріоритет рішення.
	\item Опис кроків для відтворення дефекту (неправильної поведінки програми).
	\item Відповідальний за усунення дефекту.
	\item Обговорення можливих рішень та їх наслідків.
	\item Поточний стан виправлення дефекту.
	\item Версія продукту, в якій дефект виправлений.
\end{enumerate}
Крім того, розвинені системи надають можливість прикріплювати файли, які допомагають описати проблему, наприклад, дамп пам'яті або скріншот.

Типові багтрекери підтримують концепцію життєвого циклу бага, що відстежується через статус, присвоєний багу. Багтрекер дозволяє адміністраторам конфігурувати права на основі статусу, змінювати статус бага чи вилучати баг. Система також дозволяє адміністратору конфігурувати статуси багів і до якого статусу баг може бути змінено в кожному окремому випадку. Деякі системи надсилають електронного листа зацікавленим сторонам, таким як представленому (англ. submitter) та призначеному (англ. assigned) програмістам, у разі додавання нового запису чи зміни статусу.

Типовий життєвий цикл дефекту:
\begin{enumerate}
	\item Новий — дефект зареєстрований тестувальником.
	\item Призначений — призначений відповідальний за виправлення дефекту.
	\item Дозволений — дефект переходить назад у сферу відповідальності тестувальника. Як правило, супроводжується резолюцією, наприклад:
	\begin{itemize}
		\item виправлено (виправлення включені у версію таку-то);
		\item дубль (повторює дефект, що вже знаходиться в роботі);
		\item не виправлено (працює відповідно до специфікації, має занадто низький пріоритет, виправлення відкладено до наступної версії, тощо);
		\item «в мене все працює» (запит додаткової інформації про умови, в~яких дефект проявляється).
	\end{itemize}
	\item Далі тестувальник проводить перевірку виправлення, залежно від чого дефект або знову переходить у стан «Призначений» (якщо він описаний як виправлений, але не виправлений), або у стан «Закрито».
	\item Відкрито повторно — дефект знайдено знову в іншій версії.
\end{enumerate}
Система може надавати адміністраторові можливість налаштування користувачів, які можуть переглядати і редагувати помилки залежно від їх стану, переводити їх в інший стан або видаляти.

У корпоративному середовищі, система відстеження помилок може використовуватися для отримання звітів, що показують продуктивність програмістів при виправленні помилок. Однак, часто такий підхід не дає достатньо точних результатів через те, що різні помилки мають різну ступінь серйозності та складності. При цьому серйозність проблеми прямо не стосується складності її усунення.

Головна перевага багтрекера полягає в забезпеченні чіткого централізованого огляду запитів розробки (включаючи як баги, так і зручності, різниця часто нечітка) та їх стану. Список пріоритетів незавершених пунктів (що часто називається backlog) забезпечує вагомий внесок при визначенні перспективного плану продукту, чи просто «наступного релізу».

\section{Способи та засоби реалізації багтрекерів}

Оскільки багтрекер є web-системою, то для нього актуальні всі доступні інструменти реалізації, що і для будь-якої іншої системи даного типу.

Сьогодні абсолютна більшість програмних продуктів розробляється з використанням фреймворків, оскільки незважаючи на те, що кожна предметна область має свої особливості, імплементація базових операцій низькорівневої взаємодії з системою, в рамках якої виконується додаток, є стандартною і немає сенсу займатися нею, адже вже створені імплементації не лише стабільніші, а ще й краще протестовані іншими розробниками.

Список найпопулярніших фреймворків для розробки back-end частини \cite{web_app_frameworks} на даний момент влючає 29 найменувань. Представлені фреймворки охоплюють широкий спектр мов програмування, парадигм та підходів розробки.

Серед мов програмування, підтримуваних даними фреймворками, представлені такі:
\begin{enumerate}
    \item JavaScript -- динамічна, об'єктно-орієнтована мова програмування. Реалізація стандарту ECMAScript.
    \item Ruby -- інтерпретована, повністю об'єктно-орієнтована мова програмування з чіткою динамічною типізацією.
    \item PHP -- скриптова мова програмування, була створена для генерації HTML-сторінок на стороні веб-сервера.
    \item Python -- інтерпретована об'єктно-орієнтована мова програмування високого рівня зі строгою динамічною типізацією.
    \item Java -- об'єктно-орієнтована мова програмування, випущена 1995 року компанією Sun Microsystems як основний компонент платформи Java.
    \item Scala -- мультипарадигмова мова програмування, що поєднує властивості об'єктно-орієнтованого та функційного програмування.
    \item Go -- компільована мова програмування із вбудованими засобами для паралельних обчислень і засобами віддаленого керування пакунками.
    \item C\# -- об'єктно-орієнтована мова програмування з безпечною системою типізації для платформи .NET.
    \item Perl -- високорівнева, інтерпретована, динамічна мова програмування загального призначення.
    \item Crystal -- об'єктно-орієнтована мова програмування загального призначення, що має Ruby-подібний синтаксис.
\end{enumerate}

Серед наведених вище мов програмування є як статично типізовні мови, що дозволяють ефективно розробляти високоякісні рішення бізнес-класу, так і динамично типізовані мови, що мають перевагу більш швидкого (порівняно зі статично типізованими мовами) процесу розробки та впровадження рішень.

Також серед наведеного списку є не лише інтерпретовані мови, а й компільовані. Такі мови дозволяють писати високоефективні системи, але в той же час є більш складними у використанні для розробки web-додатків.

Для покращення розуміння будови web-додатку та надання актуальної інформації щодо методів комунікації з сервером зазвичай використовують фреймворки-конструктори API. Такі фреймфорки дозволяють не лише описувати методи взаємодії с web-системою, а й створювати на основі цього опису набори моделей, котрі можна використовувати в додатках під велику кількість платформ та мов програмування, таким чином звільняючи розробників клієнтських додатків від потреби актуалізації моделей в коді в ручному режимі. Також, досить важливою перевагою виктористання фреймворку-конструктору API є можливість створення робочого каркасу додатку маючи лише правильно сформований опис структури серверного API. На даний момент найбільш продвинутим фреймворком-конструктором API є Swagger. Цей інструмент має широку розповсюдженість, відкриту ліцензію та розробляється спільнотою розробників як open source проект.

\section{Аналіз переваг і недоліків існуючих багтрекерів}

Існує велика кількість різноманітних реалізацій багтрекерів, що значно відрізняються одне від одного. Серед найбільш популярних багтрекінгових рішень, що підтримуються розробниками, переважають пропрієтарні та SaaS рішення. Велика кількість з представлених рішень, що розповсюджуються на умовах вільної ліцензії вже не підтримуються розробниками.

В таблиці \ref{table:1} наведено загальну інформацію щодо найпопулярніших існуючих багтрекінгових рішень.

\begin{tableJustNowSureWholeOnSamePage}
\footnotesize
\captionof{table}{Порівняльна таблиця загальної інформації щодо існуючих багтрекерів}
\begin{tabular}{ |p{2cm}|p{2cm}|p{2cm}|p{3.4cm}|p{3cm}|p{2cm}| } 
    \hline
    \thead{Система} &
    \thead{Автор} &
    \thead{Ліцензія} &
    \thead{Мови імплементації} &
    \thead{Джерело даних} &
    \thead{Рік запуску} \\
    \hline
    Apache Bloodhound &
    Apache Software Foundation &
    Apache License &
    Python &
    MySQL, PostgreSQL, \newline SQLite &
    2012 \\
    \hline
    Axosoft &
    Axosoft LLC &
    Proprietary, Saas &
    C\#, .NET &
    SQL Server &
    2002 \\
    \hline
    Bugzilla &
    Mozilla Foundation &
    MPL &
    Perl &
    MySQL, Oracle, \newline PostgreSQL, \newline SQLite &
    1998 \\
    \hline
    GNATS &
    Free Software Foundation &
    GPL &
    C, Perl &
    Текстові файли зі спеціальною системою пошуку &
    1992 \\
    \hline
    JIRA &
    Atlassian &
    Proprietary &
    Java &
    MySQL, PostgreSQL, \newline Oracle, SQL Server &
    2002 \\
    \hline
    Redmine &
    Jean-Philippe Lang &
    GPLv2 &
    Ruby &
    MySQL, PostgreSQL, \newline SQLite &
    2006 \\
    \hline
    Team Foundation Server &
    Microsoft &
    Proprietary, Commercial &
    .NET &
    MS SQL Server 2005 \& 2008 &
    2005 \\
    \hline
\end{tabular}
\label{table:1}
\end{tableJustNowSureWholeOnSamePage}

Оскільки створення системи, що не привносить нічого нового в контексті прикладної області не має сенсу, автори багтрекінгових систем впроваджують нові можливості в свої продукти, тим самим покращуючи якість вирішення проблеми обліку та виправлення дефектів програмного коду.

В таблиці \ref{table:2} наведено інформацію щодо особливостей існуючих багтрекінгових рішень.

\begin{tableJustNowSureWholeOnSamePage}
\footnotesize
\captionof{table}{Порівняльна таблиця особливостей існуючих багтрекерів}
\begin{tabular}{ |p{2cm}|p{2.2cm}|p{2cm}|p{2cm}|p{3cm}|p{3.2cm}| } 
    \hline
    \thead{Система} &
    \thead{Інтегрована\\документація} &
    \thead{Планування\\тестів} &
    \thead{Підтримка\\плагінів} &
    \thead{Індексований\\пошук по тексту} &
    \thead{Індексований\\пошук по файлам} \\
    \hline
    Apache Bloodhound &
    Так &
    Так &
    Так &
    Так &
    Ні \\
    \hline
    Axosoft &
    Так &
    Ні &
    Так &
    Ні &
    Ні \\
    \hline
    Bugzilla &
    Так &
    Так &
    Так &
    Так &
    Ні \\
    \hline
    GNATS &
    Ні &
    Ні &
    Ні &
    Ні &
    Ні \\
    \hline
    JIRA &
    Так &
    Так &
    Так &
    Так &
    Так \\
    \hline
    Redmine &
    Так &
    Так &
    Так &
    Так &
    Так \\
    \hline
    Team Foundation Server &
    Так &
    Так &
    Так &
    Ні &
    Ні \\
    \hline
\end{tabular}
\label{table:2}
\end{tableJustNowSureWholeOnSamePage}

За даними StatCounter \cite{statcounter_desktop_mobile}, наприкінці 2016 року кількість користувачів, що виходять в інтернет з мобільних пристроїв почала перевищувати кількість користувачів, що виходять зі стаціонарних компьютерів та ноутбуків. Такі дані свідчать про те, що основною платформою, на яку повинні орієнтуватися розробники більшості програмних продуктів публічного користування, стають мобільні платформи, такі як Android та iOS.

Серед інших методів взаємодії з користувачем важливо також враховувати такі платформи як Web та додатки для операційних систем стаціонарного використання. В професійному середовищі не завжди може бути зручно використовувати мобільний клієнт, тому наведені платформи також є досить пріоритетними для авторів багтрекінгових систем.

В таблиці \ref{table:3} наведено інформацію щодо інтерфейсів взаємодії з користувачем існуючих багтрекерів.

\begin{tableJustNowSureWholeOnSamePage}
\footnotesize
\captionof{table}{Порівняльна таблиця інтерфейсів взаємодії з користувачем існуючих багтрекерів}
\begin{tabular}{ |p{2cm}|p{2cm}|p{1.5cm}|p{2cm}|p{1.4cm}|p{2cm}|p{3cm}| } 
    \hline
    \thead{Система} &
    \thead{Web} &
    \thead{Email} &
    \thead{CLI} &
    \thead{GUI} &
    \thead{REST API} &
    \thead{Мобільний клієнт} \\
    \hline
    Apache Bloodhound &
    Так &
    Так &
    Ні &
    Ні &
    Ні &
    Ні \\
    \hline
    Axosoft &
    Так &
    Так &
    Ні &
    Так &
    Так &
    Так, для iOS \\
    \hline
    Bugzilla &
    Так &
    Так &
    Так &
    Так &
    Так &
    Ні \\
    \hline
    GNATS &
    Так &
    Так &
    Так &
    Так &
    Ні &
    Ні \\
    \hline
    JIRA &
    Так &
    Так &
    Так &
    Так &
    Так &
    Так, для iOS та Android \\
    \hline
    Redmine &
    Так &
    Так &
    Частково &
    Ні &
    Так &
    Так, для iOS та Android \\
    \hline
    Team Foundation Server &
    Так &
    Так &
    Так &
    Ні &
    Ні &
    Ні \\
    \hline
\end{tabular}
\label{table:3}
\end{tableJustNowSureWholeOnSamePage}

В наш час використання електронної пошти для отримання сповіщення щодо новин та змін в проектах є стандартним засобом, підтримуваним абсолютною більшістю систем обліку в цілому та багтрекінгових систем зокрема. Окрім електронної пошти досить популярним є також протокол RSS \cite{rss} — він дозволяє швидко оцінювати об'єм доступної інформації а також доступний на всіх відомих платформах.

В таблиці \ref{table:4} наведено інформацію щодо інтерфейсів нотифікації користувачів існуючих багтрекінгових рішень.

\begin{tableJustNowSureWholeOnSamePage}
\footnotesize
\captionof{table}{Порівняльна таблиця інтерфейсів нотифікації користувачів існуючих багтрекерів}
\begin{tabular}{ |p{2cm}|p{2cm}|p{2cm}|p{2cm}|p{2cm}|p{2cm}| } 
    \hline
    \thead{Система} &
    \thead{Email} &
    \thead{Rss} &
    \thead{Atom} &
    \thead{XMPP} &
    \thead{Twitter} \\
    \hline
    Apache Bloodhound &
    Так &
    Так &
    Ні &
    Ні &
    Ні \\
    \hline
    Axosoft &
    Так &
    Так &
    Ні &
    Ні &
    Ні \\
    \hline
    Bugzilla &
    Так &
    Так &
    Так &
    Ні &
    Ні \\
    \hline
    GNATS &
    Так &
    Ні &
    Ні &
    Ні &
    Ні \\
    \hline
    JIRA &
    Так &
    Так &
    Ні &
    Так &
    Ні \\
    \hline
    Redmine &
    Так &
    Так &
    Так &
    Так &
    Так \\
    \hline
    Team Foundation Server &
    Так &
    Ні &
    Ні &
    Ні &
    Ні \\
    \hline
\end{tabular}
\label{table:4}
\end{tableJustNowSureWholeOnSamePage}


Системи контролю версій \cite{vcs} є незамінною частиною сучасної індустрії програмної розробки. Інтеграція з такими системами надає можливість явно прив'язувати конкретну зміну програмного коду до звіту щодо певного дефекту продукту. Таким чином, маючи інтеграцію з системою контролю версій, кожен зацікавлений в усуненні дефекту член команди має можливість оцінити правильність внесених змін, а розробник, що займався усуненням проблеми позбавляється потреби прямої взаємодії з багтрекінговою системою.

В таблиці \ref{table:5} наведено інформацію щодо інтеграцій існуючих багтрекерів з системами контролю версій.

\begin{tableJustNowSureWholeOnSamePage}
\footnotesize
\captionof{table}{Порівняльна таблиця інтеграцій існуючих багтрекерів з системами контролю версій}
\begin{tabular}{ |p{2cm}|p{2cm}|p{2cm}|p{2cm}|p{2cm}|p{2cm}| } 
    \hline
    \thead{Система} &
    \thead{Git} &
    \thead{Mercurial} &
    \thead{CVS} &
    \thead{Subversion} &
    \thead{Perforce} \\
    \hline
    Apache Bloodhound &
    Так &
    Так &
    Так &
    Так &
    Так \\
    \hline
    Axosoft &
    Так &
    Так &
    Ні &
    Так &
    Так \\
    \hline
    Bugzilla &
    Так &
    Так &
    Так &
    Так &
    Так \\
    \hline
    GNATS &
    Ні &
    Ні &
    Так &
    Ні &
    Ні \\
    \hline
    JIRA &
    Так &
    Так &
    Так &
    Так &
    Так \\
    \hline
    Redmine &
    Так &
    Так &
    Так &
    Так &
    Так \\
    \hline
    Team Foundation Server &
    Так &
    Ні &
    Ні &
    Так &
    Ні \\
    \hline
\end{tabular}
\label{table:5}
\end{tableJustNowSureWholeOnSamePage}

Серед засобів аутентифікації найбільш популярними є такі:
\begin{enumerate}
	\item Form based -- найпростіший метод аутентифікації, що вимагає від користувача надання пари логін-пароль.
	\item Public key cryptography -- метод аутентифікації, що полягає у використанні пари приватний ключ--публічний ключ, а також сертифікаційного центру, що забезпечує безпечну взаємодію з приватним ключем.
	\item Two-factor -- метод, що полягає у створенні додаткового кроку аутентифікації. В ролі додаткового кроку зазвичай використовують перехід за посиланням або введеня коду надісланих на заздалегідь прив'язаний до аккаунту користувача номер телефону або E-Mail. Такий метод є більш безпечним ніж усі попередні, оскільки для отримання доступу до аккаунта користувача хакеру буде потрібно взнати не лише пароль від його облікового запису в межах даного сервісу, а ще й отримати доступ до акаунту в іншому сервісі або специфічного пристрою.
	\item OpenID -- децентралізована система єдиного входу, котра дозволяє використовувати один обліковий запис на великій кількості сайтів завдяки делегації авторизації спеціальному "провайдеру ідентифікації".
	\item OAuth -- відкритий стандарт авторизації, який дозволяє користувачам відкривати доступ до своїх приватних даних (фотографії, відео, списки контактів), що зберігаються на одному сайті, іншому сайту, без необхідності вводу імені користувача та паролю.
\end{enumerate}

В таблиці \ref{table:6} наведено інформацію щодо методів аутентифікації в існуючих багтрекерах.

\begin{center}
\footnotesize
\captionof{table}{Порівняльна таблиця методів аутентифікації в існуючих багтрекерах}
\begin{tabular}{ |p{2cm}|p{2cm}|p{2.5cm}|p{2cm}|p{2cm}|p{1.5cm}| } 
    \hline
    \thead{Система} &
    \thead{Form based} &
    \thead{Public key\\cryptography} &
    \thead{Two-factor} &
    \thead{OpenID} &
    \thead{OAuth} \\
    \hline
    Apache Bloodhound &
    Так &
    Так &
    Ні &
    Так &
    Ні \\
    \hline
    Axosoft &
    Так &
    Ні &
    Ні &
    Так &
    Ні \\
    \hline
    Bugzilla &
    Так &
    Ні &
    Ні &
    Так &
    Ні \\
    \hline
    GNATS &
    Ні &
    Ні &
    Ні &
    Ні &
    Ні \\
    \hline
    JIRA &
    Так &
    Ні &
    Ні &
    Так &
    Так \\
    \hline
    Redmine &
    Так &
    Так &
    Ні &
    Так &
    Ні \\
    \hline
    Team Foundation Server &
    Так &
    Так &
    Ні &
    Ні &
    Ні \\
    \hline
\end{tabular}
\label{table:6}
\end{center}


\section{Задача розробки}

На данний момент, ринок багтрекерів заповнений комерційними реалізаціями з пропрієтарними ліцензіями на використання, що створює фінансові труднощі в інтеграції існуючих рішень. Також існує проблема недостатньої оптимізації існуючих рішень під мобільні платформи та сценарії портативного використання.

Щоб відповідати вимогам сучасного світу розробки ПЗ, майбутній продукт повинен задовольняти такі вимоги:
\begin{enumerate}
    \item Розповсюджуватись на основі вільної ліцензії.
    \item Мати можливість швидкого розгортання на сервері користувача.
    \item Надавати API для взаємодії з клієнтами під будь-яку платформу.
    \item Мати готову реалізацію клієнту під платформу Android.
    \item Надавати можливість створювати/редагувати/видаляти звіти щодо багів/пропозицій.
    \item Надавати можливість відстежувати зміни стану багів/пропозицій.
    \item Надавати можливість призначати відповідального за баг/пропозицію.
    \item Мати систему керування ролями користувачів в межах організації.
\end{enumerate}

Практична цінністю майбутнього продукту є надання комплексу ПЗ для інтеграції багтрекінгової системи в корпоративні системи клієнтів на безкоштовній основі, з відкритою кодовою базою.

\end{document}