\documentclass[../main.tex]{subfiles}

\begin{document}

\chapter{Теоретичні відомості та аналітика}

\section{Призначення та область застосування багтрекерів}

Розробка будь-якої системи є трудомістким та важким процесом, що супроводжується високим ризиком допущення помилки на всіх етапах. Тому, зазвичай, паралельно в розробкою системи відбувається і її тестування на предмет нових дефектів та регресій.

Для того щоб дефект був усунутий, інформація про нього повинна дістатися до розробника, котрий зможе модифікувати код відповідним чином.

Система відстеження помилок у найпростішому варіанті — це процес, що включає в себе виявлення помилки, її опис, виправлення і перевірку цього виправлення, тобто процес «стеження» за багом протягом всього як його життєвого циклу, так і життєвого циклу розробки в цілому.\cite{bugtracking_systems}

Багтрекери є підвидом таск-менеджерів. Основною відмінністю багтрекерів від таск-менеджерів є  оріентованість на покращення ефективності вирішення проблем, що зазвичай виявляється в наданні додатковогу функціоналу, пов'язаного з прикладною областю. Наприклад, для будь-якого програмного продукту досить важливим засобом пошуку проблеми є stack trace помилки, тому наявність підтримки прикріплення stack trace'у до помилки є досить суттєвим покращенням порівняно зі звичайним таск-менеджером щодо аналізу проблеми, що виникла під час роботи продукту. Для ілюстрації можливого засобу поліпшення ефективності вирішення проблем в конкретній прикладній області програмуваня можна навести приклад з застосунком, основною задачею котрого є відстеження локацій користувача задля передбачення можливих перешкод на шляху руху автомобіля - у випадку цієї прикладної області було б доцільно мати можливість прикріплення звіту щодо отриманих географічних даних з самого застосунку до основного звіту щодо проблеми, оскільки ця інформація є необхідною для вирішення всіх проблем, пов'язаних з невірним трактуванням поточних доріг, прилеглих перешкод, тощо.

Типові багтрекери підтримують концепцію життєвого циклу бага, що відстежується через статус, присвоєний багу. Багтрекер дозволяє адміністраторам конфігурувати права на основі статусу, змінювати статус бага чи вилучати баг. Система також дозволяє адміністратору конфігурувати статуси багів і до якого статусу баг може бути змінено в кожному окремому випадку. Деякі системи надсилають електронного листа зацікавленим сторонам, таким як представленому (англ. submitter) та призначеному (англ. assigned) програмістам, у разі додавання нового запису чи зміни статус.

Головна перевага багтрекера полягає в забезпеченні чіткого централізованого огляду запитів розробки (включаючи як баги, так і зручності, різниця часто нечітка) та їх стану. Список пріоритетів незавершених пунктів (що часто називається backlog) забезпечує вагомий внесок при визначенні перспективного плану продукту, чи просто «наступного релізу».

У корпоративному середовищі багтрекер може використовуватися для генерації звітів із продуктивності програмістів у виправленні багів. Однак, це може інколи спричинити неточний результат через те, що різні баги можуть мати різні рівні серйозності й складності. Серйозність бага не може безпосередньо пов'язуватись зі складністю його виправлення. Погляди менеджерів та архітекторів можуть відрізнятись.

\section{Способи та засоби реалізації багтрекерів}

Оскільки багтрекер є web-системою, то для нього актуальні всі доступні інструменти реалізації, що і для будь-якої іншої системи даного типу.

Сьогодні абсолютна більшість програмних продуктів розробляється з використанням фреймворків, оскільки кожна предметна область має свої 

Список найпопулярніших фреймворків для розробки back-end частини \cite{web_app_frameworks} на даний момент влючає 29 найменувань. Представлені фреймворки охоплюють широкий спектр мов програмування, парадигм та підходів розробки.

Серед мов програмування програмування представлені такі:
\begin{itemize}
    \item JavaScript
    \item Ruby
    \item PHP
    \item Python
    \item Java
    \item Scala
    \item Go
    \item C\#
    \item Perl
    \item Crystal
\end{itemize}

Серед наведених вище мов програмування є як статично типізовні мови, що дозволяють ефективно розробляти високоякісні рішення бізнес-класу, так і динамично типізовані мови, що мають перевагу більш швидкого (порівняно зі статично типізованими мовами) процесу розробки та впровадження рішень.

Також серед наведеного списку є не лише інтерпритовані мови, а й компільовані. Такі мови дозволяють писати високоефективні системи, але в той же час є більш складними у використанні для розробки web-додатків.

Для покращення розуміння будови web-додатку та надання актуальної інформації щодо методів комунікації з сервером зазвичай використовують фреймворки-конструктори API. Такі фреймфорки дозволяють не лише описувати методи взаємодії с web-системою, а й створювати на основі цього опису набори моделей, котрі можна використовувати в додатках під велику кількість платформ та мов програмування, таким чином звільняючи розробників клієнтських додатків від потреби актуалізації моделей в коді в ручному режимі. Також, досить важливою перевагою виктористання фреймворку-конструктору API є можливість створення робочого каркасу додатку маючи лише правильно сформований опис структури серверного API. На даний момент найбільш продвинутим фреймворком-конструктором API є Swagger. Цей інструмент має широку розповсюдженість, відкриту ліцензію та розробляється спільнотою розробників як open source проект.

\section{Аналіз переваг і недоліків існуючих багтрекерів}

Існує велика кількість різноманітних реалізацій багтрекерів, що значно відрізняються одне від одного. Серед найбільш популярних багтрекінгових рішень, що підтримуються розробниками, переважають пропрієтарні та SaaS рішення. Велика кількість з представлених рішень, що розповсюджуються на умовах вільної ліцензії вже не підтримуються розробниками.

В таблиці \ref{table:1} наведено загальну інформацію щодо найпопулярніших існуючих багтрекінгових рішень.

\begin{center}
\footnotesize
\captionof{table}{Порівняльна таблиця загальної інформації щодо існуючих багтрекерів}
\begin{tabular}{ |p{2cm}|p{2cm}|p{2cm}|p{4cm}|p{3cm}|p{2cm}| } 
    \hline
    \thead{Система} &
    \thead{Автор} &
    \thead{Ліцензія} &
    \thead{Мови імплементації} &
    \thead{Джерело даних} &
    \thead{Рік запуску} \\
    \hline
    Apache Bloodhound &
    Apache Software Foundation &
    Apache License &
    Python &
    MySQL, PostgreSQL, \newline SQLite &
    2012 \\
    \hline
    Axosoft &
    Axosoft LLC &
    Proprietary, Saas &
    C\#, .NET &
    SQL Server &
    2002 \\
    \hline
    Bugzilla &
    Mozilla Foundation &
    MPL &
    Perl &
    MySQL, Oracle, \newline PostgreSQL, \newline SQLite &
    1998 \\
    \hline
    GNATS &
    Free Software Foundation &
    GPL &
    C, Perl &
    Текстові файли зі спеціальною системою пошуку &
    1992 \\
    \hline
    JIRA &
    Atlassian &
    Proprietary &
    Java &
    MySQL, PostgreSQL, \newline Oracle, SQL Server &
    2002 \\
    \hline
    Redmine &
    Jean-Philippe Lang &
    GPLv2 &
    Ruby &
    MySQL, PostgreSQL, \newline SQLite &
    2006 \\
    \hline
    Team Foundation Server &
    Microsoft &
    Proprietary, Commercial &
    .NET &
    MS SQL Server 2005 \& 2008 &
    2005 \\
    \hline
\end{tabular}
\label{table:1}
\end{center}

В таблиці \ref{table:2} наведено інформацію щодо особливостей існуючих багтрекінгових рішень.

\begin{center}
\footnotesize
\captionof{table}{Порівняльна таблиця особливостей існуючих багтрекерів}
\begin{tabular}{ |p{2cm}|p{2.8cm}|p{2cm}|p{2cm}|p{3cm}|p{3.2cm}| } 
    \hline
    \thead{Система} &
    \thead{Інтегрована\\документація} &
    \thead{Планування\\тестів} &
    \thead{Підтримка\\плагінів} &
    \thead{Індексований\\пошук по тексту} &
    \thead{Індексований\\пошук по файлам} \\
    \hline
    Apache Bloodhound &
    Так &
    Так &
    Так &
    Так &
    Ні \\
    \hline
    Axosoft &
    Так &
    Ні &
    Так &
    Ні &
    Ні \\
    \hline
    Bugzilla &
    Так &
    Так &
    Так &
    Так &
    Ні \\
    \hline
    GNATS &
    Ні &
    Ні &
    Ні &
    Ні &
    Ні \\
    \hline
    JIRA &
    Так &
    Так &
    Так &
    Так &
    Так \\
    \hline
    Redmine &
    Так &
    Так &
    Так &
    Так &
    Так \\
    \hline
    Team Foundation Server &
    Так &
    Так &
    Так &
    Ні &
    Ні \\
    \hline
\end{tabular}
\label{table:2}
\end{center}

В таблиці \ref{table:3} наведено інформацію щодо особливостей існуючих багтрекінгових рішень.

\begin{center}
\footnotesize
\captionof{table}{Порівняльна таблиця інтерфейсів взаємодії з користувачем існуючих багтрекерів}
\begin{tabular}{ |p{2cm}|p{2cm}|p{1.5cm}|p{2cm}|p{2cm}|p{2cm}|p{3cm}| } 
    \hline
    \thead{Система} &
    \thead{Web} &
    \thead{Email} &
    \thead{CLI} &
    \thead{GUI} &
    \thead{REST API} &
    \thead{Мобільний клієнт} \\
    \hline
    Apache Bloodhound &
    Так &
    Так &
    Ні &
    Ні &
    Ні &
    Ні \\
    \hline
    Axosoft &
    Так &
    Так &
    Ні &
    Так &
    Так &
    Так, для iOS \\
    \hline
    Bugzilla &
    Так &
    Так &
    Так &
    Так &
    Так &
    Ні \\
    \hline
    GNATS &
    Так &
    Так &
    Так &
    Так &
    Ні &
    Ні \\
    \hline
    JIRA &
    Так &
    Так &
    Так &
    Так &
    Так &
    Так, для iOS та Android \\
    \hline
    Redmine &
    Так &
    Так &
    Частково &
    Ні &
    Так &
    Так, для iOS та Android \\
    \hline
    Team Foundation Server &
    Так &
    Так &
    Так &
    Ні &
    Ні &
    Ні \\
    \hline
\end{tabular}
\label{table:3}
\end{center}

В таблиці \ref{table:4} наведено інформацію щодо інтерфейсів нотифікації користувачів існуючих багтрекінгових рішень.

\begin{center}
\footnotesize
\captionof{table}{Порівняльна таблиця інтерфейсів нотифікації користувачів існуючих багтрекерів}
\begin{tabular}{ |p{2cm}|p{2cm}|p{2cm}|p{2cm}|p{2cm}|p{2cm}| } 
    \hline
    \thead{Система} &
    \thead{Email} &
    \thead{Rss} &
    \thead{Atom} &
    \thead{XMPP} &
    \thead{Twitter} \\
    \hline
    Apache Bloodhound &
    Так &
    Так &
    Ні &
    Ні &
    Ні \\
    \hline
    Axosoft &
    Так &
    Так &
    Ні &
    Ні &
    Ні \\
    \hline
    Bugzilla &
    Так &
    Так &
    Так &
    Ні &
    Ні \\
    \hline
    GNATS &
    Так &
    Ні &
    Ні &
    Ні &
    Ні \\
    \hline
    JIRA &
    Так &
    Так &
    Ні &
    Так &
    Ні \\
    \hline
    Redmine &
    Так &
    Так &
    Так &
    Так &
    Так \\
    \hline
    Team Foundation Server &
    Так &
    Ні &
    Ні &
    Ні &
    Ні \\
    \hline
\end{tabular}
\label{table:4}
\end{center}

В таблиці \ref{table:5} наведено інформацію щодо інтеграцій існуючих багтрекерів з системами контролю версій.

\begin{center}
\footnotesize
\captionof{table}{Порівняльна таблиця інтеграцій існуючих багтрекерів з системами контролю версій}
\begin{tabular}{ |p{2cm}|p{2cm}|p{2cm}|p{2cm}|p{2cm}|p{2cm}| } 
    \hline
    \thead{Система} &
    \thead{Git} &
    \thead{Mercurial} &
    \thead{CVS} &
    \thead{Subversion} &
    \thead{Perforce} \\
    \hline
    Apache Bloodhound &
    Так &
    Так &
    Так &
    Так &
    Так \\
    \hline
    Axosoft &
    Так &
    Так &
    Ні &
    Так &
    Так \\
    \hline
    Bugzilla &
    Так &
    Так &
    Так &
    Так &
    Так \\
    \hline
    GNATS &
    Ні &
    Ні &
    Так &
    Ні &
    Ні \\
    \hline
    JIRA &
    Так &
    Так &
    Так &
    Так &
    Так \\
    \hline
    Redmine &
    Так &
    Так &
    Так &
    Так &
    Так \\
    \hline
    Team Foundation Server &
    Так &
    Ні &
    Ні &
    Так &
    Ні \\
    \hline
\end{tabular}
\label{table:5}
\end{center}

В таблиці \ref{table:6} наведено інформацію щодо методів аутентифіфкації в існуючих багтрекерах.

\begin{center}
\footnotesize
\captionof{table}{Порівняльна таблиця методів аутентифіфкації в існуючих багтрекерах}
\begin{tabular}{ |p{2cm}|p{2cm}|p{2.5cm}|p{2cm}|p{2cm}|p{1.5cm}| } 
    \hline
    \thead{Система} &
    \thead{Form based} &
    \thead{Public key\\cryptography} &
    \thead{Two-factor} &
    \thead{OpenID} &
    \thead{OAuth} \\
    \hline
    Apache Bloodhound &
    Так &
    Так &
    Ні &
    Так &
    Ні \\
    \hline
    Axosoft &
    Так &
    Ні &
    Ні &
    Так &
    Ні \\
    \hline
    Bugzilla &
    Так &
    Ні &
    Ні &
    Так &
    Ні \\
    \hline
    GNATS &
    Ні &
    Ні &
    Ні &
    Ні &
    Ні \\
    \hline
    JIRA &
    Так &
    Ні &
    Ні &
    Так &
    Так \\
    \hline
    Redmine &
    Так &
    Так &
    Ні &
    Так &
    Ні \\
    \hline
    Team Foundation Server &
    Так &
    Так &
    Ні &
    Ні &
    Ні \\
    \hline
\end{tabular}
\label{table:6}
\end{center}

\section{Задача розробки}

На данний момент, ринок багтрекерів заповнений комерційними реалізаціями з пропрієтарними ліцензіями на використання, що створює фінансові труднощі в інтеграції існуючих рішень. Також існує проблема недостатньої оптимізації існуючих рішень під мобільні платформи та сценарії портативного використання.

Щоб відповідати вимогам сучасного світу розробки ПЗ, майбутній продукт повинен задовольняти такі вимоги:
\begin{itemize}
    \item Розповсюджуватись на основі вільної ліцензії
    \item Мати можливість швидкого розгортання на сервері користувача
    \item Надавати API для взаемодії з клієнтами під бюдь-яку платформу
    \item Мати готову реалізацію клієнту під платформу Android
    \item Надавати можливість створювати/редагувати/видаляти звіти щодо багів/пропозицій
    \item Надавати можливість відстежувати зміни стану багів/пропозицій
    \item Надавати можливість назначати відповідального за баг/пропозицію
    \item Мати систему керування ролями користувачів в межах організації
\end{itemize}

Практична цінністю майбутнього продукту є надання комплексу ПЗ для інтеграції багтрекінгової системи в корпоративні системи клінтів на безкоштовній основі, з відкритою кодовою базою.

\end{document}