\documentclass[../main.tex]{subfiles}

\begin{document}

\chapter*{ВИСНОВКИ}\addcontentsline{toc}{chapter}{ВИСНОВКИ}

Багтрекери використовуються більшістю проектів програмногного забезпечення. Такі системи використовуть не лише великі компанії, а й окремі розробники для ведення обліку проблем, що потребують рішення. Немає жодної open-source організції, проекти якої не мали б свого багтрекеру. Така популярність багтрекерів зумовлена, в першу чергу, підвищенням якості кінцевого продукту завдяки системам нотифікації, типізації та пріоретизації проблем.

Будь-який з існуючих багтрекерів з~більшим або меншим успіхом вирішує проблему обліку багів, але в області багтрекерів існує проблема недостатньої адаптації вільно-розповсюджуваних багтрекінгових систем під сучасні платформи. Саме ця проблема була одним з найголовніших факторів, що вплинули на рішення про необхідність розробки нової системи.

Для того, щоб подолати проблему адаптації під різні платформи та інші проблеми існуючих рішень, до системи було висунуто наступні вимоги:

\begin{enumerate}
	\item Розповсюдження на основі вільної ліцензії.
	\item Можливість швидкого розгортання на сервері користувача.
	\item Надання API для взаемодії з клієнтами під будь-яку платформу.
	\item Наявність готової реалізації клієнту під платформу Android.
	\item Можливість створювати/редагувати/видаляти звіти щодо багів/пропозицій.
	\item Можливість відстежувати зміни стану багів/пропозицій.
	\item Можливість назначати відповідального за баг/пропозицію.
	\item Наявність системи керування ролями користувачів в межах проекту.
\end{enumerate}

Кожну з поставлених вимог було виконано.

\subparagraph{Розповсюдження на основі вільної ліцензії.}
Розроблений програмний продукт (як серверна так і клієнтська частини) доступний абсолютно безкоштовно для будь-яких цілей, має відкритий вихідний код, та розповсюджуеться на основі ліцензії MIT.

\subparagraph{Можливість швидкого розгортання на сервері користувача.}
Однією з важливих вимог клієнт-серверних систем є простота розгортання системи на цільових машинах. Цю вимогу було задоволено завдяки вибору Node.js в ролі рушія серверу. Таким чином, система може бути розгорнутою в три простих кроки:
%Однією з важливих вимог клієнт-серверних систем є простота розгортання системи на цільових машинах. Цю вимогу було задоволено завдяки вибору Node.js в ролі рушія серверу, та наявності можливості переключення цільової СУБД. Таким чином, система може бути розгорнутою в три простих кроки:
\begin{enumerate}
	\item Встановити на машину Node.js
	\item Завантажити файли серверу
	\item Запустити з кореневого каталогу команду ''npm install \&\& node app.js''
\end{enumerate}

\subparagraph{Надання API для взаемодії з клієнтами під будь-яку платформу.}
Одною з основних цілей даного проекту було створення системи, для якої написання клієнського додатку під нову платформу було б якомога простішим. Наявність актуальної формальної специфікації методів взаємодії з сервером з використанням Swagger вирішує цю проблему. Кожен бажаючий, що має деякий багаж знань клієнтської платформи з легкістю зможе написати клієнтський додаток отриманої системи, адже абсолютно всі дані, що йому потрібні для взаємодії з сервером знаходяться в одному файлі формальної специфікації. Більш того, для кожного методу взаємодії Swagger вміє генерувати коректні приклади запитів, тому у розробника клієнтської частини не буде виникати проблем з тим, що деякі дані щодо параметрів або структури запитів йому не доступні.

\subparagraph{Наявність готової реалізації клієнту під платформу Android.}
Існування готового Android клієнту обумовлено двома причинами:
\begin{enumerate}
	\item Даний клієнтський додаток дозволяє продемонструвати можливості системи
	\item Може виступати в ролі прикладу для реалізації клієнтів під інші платформи.
\end{enumerate}

\subparagraph{Можливість створювати/редагувати/видаляти звіти щодо багів/пропозицій.}
Дана можливість є основою будь-якого багтрекеру. Методи взаємодії, що необхідні дня задоволення даної вимоги наявні (див. рис. \ref{available_rest_endpoints}). Приклади користувацього інтерфейсу з відображенням даних щодо звіту доступні в додатках \ref{scr_ui_issue_1}, \ref{scr_ui_issue_2}, \ref{scr_ui_issue_edit}.

\subparagraph{Можливість відстежувати зміни стану багів/пропозицій.}
Дана можливість забезпечується завдяки push- та email- нотифікаціям.

\subparagraph{Можливість назначати відповідального за баг/пропозицію.}
Дана можливість є необхідною для функціонування багтрекеру в умовах великої кількості учасників проекту. Вона забезпечена наявністю таблці Issue Assignments (див. рис. \ref{db_physical}) та відповідних методів взаємодії з сервером (див. рис. \ref{available_rest_endpoints}).

\subparagraph{Наявність системи керування ролями користувачів в межах проекту.}
Дана можливість також є необхідною для функціонування багтрекеру в умовах великої кількості учасників проекту. Вона забезпечена наявністю таблці ''Project Member Roles'' (див. рис. \ref{db_physical}) та відповідних методів взаємодії з сервером (див. рис. \ref{available_rest_endpoints}).

Розвиток системи планується проводити і далі. У майбутньому система отримає можливість генерації картинки попереднього перегляду для текстових файлів і підтримку популярних сервісів хостингу git репозиторіїв.

%Розвиток системи планується проводити і далі. У майбутньому система отримає можливість створення підпроектів та звітів, що відносяться одразу до декількох проектів, також планується можливість генерації картинки попереднього перегляду для текстових файлів і підтримка популярних сервісів хостингу git репозиторіїв.

\end{document}