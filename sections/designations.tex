\documentclass[../main.tex]{subfiles}

\begin{document}

% \theoremstyle{definition}
% \begin{definition}{Програмний фреймворк}
% готовий до використання комплекс програмних рішень, включаючи дизайн, логіку та базову функціональність системи або підсистеми. Відповідно — програмний фреймворк може містити в собі також допоміжні програми, деякі бібліотеки коду, скрипти та загалом все, що полегшує створення та поєднання різних компонентів великого програмного забезпечення чи швидке створення готового і не обов'язково об'ємного програмного продукту. Побудова кінцевого продукту відбувається, зазвичай, на базі єдиного API.
% \end{definition}

\begin{itemize}
    \item Баг - дефект програмного забезпечення
    \item Багтрекер - система обліку багів
    \item Таск-менеджер - система обліку задач
    \item Stack trace - звіт про активні стекові кадри (активаційні записи) в певній точці програми під час її виконання.
    \item Програмний фреймворк (англ. software framework) — це готовий до використання комплекс програмних рішень, включаючи дизайн, логіку та базову функціональність системи або підсистеми. Відповідно — програмний фреймворк може містити в собі також допоміжні програми, деякі бібліотеки коду, скрипти та загалом все, що полегшує створення та поєднання різних компонентів великого програмного забезпечення чи швидке створення готового і не обов'язково об'ємного програмного продукту. Побудова кінцевого продукту відбувається, зазвичай, на базі єдиного API.
\end{itemize}

\end{document}